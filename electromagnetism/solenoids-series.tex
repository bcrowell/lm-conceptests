\begin{qu}[A solomonic solenoid]\num
The inductance of an ideal solenoid is $L=(4\pi k/c^2)N^2A/\ell$, where $N$ is
the number of turns, $A$ is the cross-sectional area, and $\ell$ is the length.
Inductance is defined by $U=(1/2)LI^2$.
Suppose that a solenoid has values of the variables $B$, $N$, $I$, $U$, and $L$.
We now chop it in half along a line perpendicular to its axis, forming two smaller,
equal solenoids, and we connect these two smaller solenoids in series. Then each
small solenoid has values ---

\vspace{8mm}

\begin{tabular}{llllll}
A.  &   $B/2$,  &   $N/2$,  &   $I/2$,  &   $U/4$,  &   and $L/4$ \\
B.  &   $B$,  &   $N/2$,  &   $I$,  &   $U/2$,  &   and $L/2$ \\
C.  &   $B/2$,  &   $N/2$,  &   $I/2$,  &   $U/2$,  &   and $L/2$ \\
D.  &   $2B$,  &   $N$,  &   $I$,  &   $U$,  &   and $L/2$ 
\end{tabular}

\end{qu}
